% (c) 2012, Jan-Piet Mens <jpmens () gmail.com>
%
% This file is part of Ansible
%
% Ansible is free software: you can redistribute it and/or modify
% it under the terms of the GNU General Public License as published by
% the Free Software Foundation, either version 3 of the License, or
% (at your option) any later version.
%
% Ansible is distributed in the hope that it will be useful,
% but WITHOUT ANY WARRANTY; without even the implied warranty of
% MERCHANTABILITY or FITNESS FOR A PARTICULAR PURPOSE.  See the
% GNU General Public License for more details.
%
% You should have received a copy of the GNU General Public License
% along with Ansible.  If not, see <http://www.gnu.org/licenses/>.
%
% This LaTeX style is very strongly copied with permission from the GRML-ZSH-Refcard
% project at https://github.com/grml/grml-gen-zshrefcard
%
% Other portions (c)Jan-Piet Mens and taken from "Alternative DNS Servers"

\documentclass[10pt,             % 8pt
               a4paper,         % A4
               oneside,         % Einseitig
               DIV20,           % Papiergr��e
             % DIV15,           % Gr��er
             % draft,           % Entwurf
               headsepline,     % Trennlinie oben
               footsepline,     % -""-       unten
               smallheadings,   % Kleine �berschriften
             % pointlessnumbers,% Keine Punkte
               halfparskip,     % Halbe Zeile Absatz statt Einzug
               nochapterprefix, % Kein "Kapitel"
             % bibtotoc         % "Literatur" im TOC  oder
             % bibtotocnumbered,% -""-, nummeriert
             % idxtotoc,        % Index im TOC
               twocolumn
              ]{scrartcl}

%%%%%%%%%%%%%%%%%%%%%%%%%%%%%%%%%%%%%%%%%%%%%%%%%%%%%%%%%%%%
%%% Pakete {{{
\usepackage[latin1]{inputenc}       % ISO-Umlaute
\usepackage[T1]{fontenc}            % T1-kodierte Fonts
%\usepackage{ae,aecompl}             % Kodierung f�r PDF
%\usepackage{ngerman}                % Deutsche Trennungen,
                                    % dt. Begriffe
\usepackage{setspace}               % Single- oder Onehalfspacing
%\setcounter{tocdepth}{4}            % 4 Hirarchien im Inhaltsv.
\usepackage{times}                  % Times als Schrift
%\usepackage{amsmath,amssymb,amstext}% Mathematische Symbole
\usepackage{url}                    % Darstellung von URLs
%\usepackage{calc}
\usepackage[iso,german]{isodate}
\usepackage{fancyvrb}

%%% Optional, je nach Dokument
% \usepackage{listings}             % Quelltext-Listings
% \usepackage{units}                % Technische Units
% \usepackage{psfrag}               % Ersetzts PS-Schriften
\usepackage{color}                % Farben in LaTeX
% \usepackage{floatflt}             % Textumflossene Bilder...
% \usepackage{picins}               % Textumflossene Bilder
% \usepackage{textcomp}             % Spezielle Zeichen
% \usepackage[small,compact]{titlesec}             % �berschriften mit wenig Platz
% \usepackage{gensymb}              % Spezielle Zeichen
% \usepackage{eurosym}              % Euro-Symbol

%%% Layout
\usepackage{scrpage2}               % KOMA-�berschriften und -Fu�zeilen.
%%% }}}
%%%%%%%%%%%%%%%%%%%%%%%%%%%%%%%%%%%%%%%%%%%%%%%%%%%%%%%%%%%%

\usepackage[pdftex]{graphicx}
\DeclareGraphicsExtensions{.pdf}
\pdfcompresslevel=9
\definecolor{light-gray}{gray}{0.1}%JPMENS
\usepackage[%
  pdftex=true,
  backref=true,
  colorlinks=true,
  bookmarks=true,
  breaklinks=true,
  linktocpage=true,
  bookmarksopen=false,
  bookmarksnumbered=false,
  pdftitle={Ansible Booklet},
  pdfsubject={Ansible},
  pdfauthor={Jan-Piet Mens},
  pdfkeywords={ansible, configuration, management, deployment, Python, SSH},
  pdfpagemode=None,
  linkcolor=black,
  urlcolor=light-gray
]{hyperref}
%%% }}}
%%%%%%%%%%%%%%%%%%%%%%%%%%%%%%%%%%%%%%%%%%%%%%%%%%%%%%%%%%%%

%%%%%%%%%%%%%%%%%%%%%%%%%%%%%%%%%%%%%%%%%%%%%%%%%%%%%%%%%%%%
%%% Eigene Funktionen {{{
%%% Beispiel:  \bild{200pt}{foo}{That's a foo\ldots}
\newcommand{\bild}[4]{
  \begin{figure}[ht]
    \includegraphics[width=#1, keepaspectratio=true]{#2}
    \caption{#3}
    \label{#4}
  \end{figure}
}

\newcommand{\kbd}[1]{\texttt{#1}}
\newcommand{\commandlistbegin}{
  \vspace{3pt}
  \begin{tabular}{ll}
}
\newcommand{\commandlistend}{
  \end{tabular}
}
\newcommand{\command}[2]{
  \texttt{#1} & \quad #2 \\
}
% -- Begin JPMENS
\newcommand{\I}[1]{\textit{#1}}%
\newcommand{\B}[1]{\textbf{#1}}%
\newcommand{\C}[1]{\texttt{#1}}%
\usepackage{color}
\newcommand{\M}[1]{\C{#1}}
\newcommand{\master}{\I{master}}%
\newcommand{\nodes}{\I{nodes}}%
\newcommand{\node}{\I{node}}%

\newcommand{\modulelistbegin}{
  \vspace{8pt}
  \begin{tabular}{l p{1.8cm} l}
}
\newcommand{\modulelistend}{
  \end{tabular}
  \vspace{8pt}
}
\newcommand{\module}[3]{
  #1 \quad & \C{\relsize{-1}#2} & \quad #3 \\
}
%% TEST
\newcommand{\man}{$\bullet$}
\newcommand{\opt}{$\circ$}
\newcommand{\ansible}{\I{Ansible}}%

%%\newcommand{\modbeginhead}{
%%  \vspace{6pt}
%%  \begin{tabular}{p{4.3cm}p{6cm}}
%%}
%%\newcommand{\modendhead}{
%%  \end{tabular}
%%}
%%\newcommand{\modhead}[3]{
%%  % section name, url completer, description
%%  \C{\B{\relsize{+1}\href{http://ansible.github.com/modules.html\##2}{#1}}} & \relsize{-1}{#3} \\
%%}

%@@\usepackage{titlesec}
%@@\titleformat{\paragraph}[wrap]
%@@  {\normalfont\fontseries{b}\selectfont\filright}
%@@  {\thesection.}{.5em}{}
%@@\titlespacing{\paragraph}
%@@  {12pc}{1.5ex plus .1ex minus .2ex}{2pc}

\newcommand{\mods}[3]{% MODSECTION
  % section name, url completer, description
  % \paragraph{\C{\B{\relsize{+1}\href{http://ansible.github.com/modules.html\##2}{#1}}}} {#3}
  \subsubsection*{\C{\B{\relsize{+1}\href{http://ansible.github.com/modules.html\##2}{#1}}}}
  {#3}
}

\newenvironment{xlist}[1]
 {\begin{list}{}%
  {\renewcommand\makelabel[1]{{##1}\hfil}%
   \settowidth{\labelwidth}{#1}%
   \setlength{\labelsep}{0.5cm}%
   \setlength{\leftmargin}{\labelwidth}%
   \addtolength{\leftmargin}{\labelsep}%
   \setlength{\rightmargin}{0pt}%
   \setlength{\parsep}{0.5ex plus .2ex minus0.1ex}%
   \setlength{\itemsep}{0ex plus0.2ex}}}%
{\end{list}}

\DefineVerbatimEnvironment{code}{Verbatim}{fontsize=\small}
\usepackage{relsize}
\DefineVerbatimEnvironment{extymeta}{Verbatim}{fontfamily=courier,tabsize=0,xleftmargin=0.0cm,commandchars=\\\{\},fontsize=\relsize{-1}}

%% -- End JPMENS

%%% }}}
%%%%%%%%%%%%%%%%%%%%%%%%%%%%%%%%%%%%%%%%%%%%%%%%%%%%%%%%%%%%

%%%%%%%%%%%%%%%%%%%%%%%%%%%%%%%%%%%%%%%%%%%%%%%%%%%%%%%%%%%%
%%% Pagestyle {{{
  \pagestyle{scrheadings}
% \pagestyle{fancyhdrs}
% \pagestyle{empty}
%%% }}}
%%%%%%%%%%%%%%%%%%%%%%%%%%%%%%%%%%%%%%%%%%%%%%%%%%%%%%%%%%%%

%%%%%%%%%%%%%%%%%%%%%%%%%%%%%%%%%%%%%%%%%%%%%%%%%%%%%%%%%%%%
%%% Seitenkopf- und -Fu�zeilen {{{
 \automark[subsection*]{section} % \left- und \rightmark bekommen Inhalt
%%% Oben: Links, Mitte, Rechts
 \ihead[]{{\Huge Ansible-Booklet}}
 \chead[]{}
 \ohead[]{\small\C{\{"pushed"\,:\,"\today{}"\}}}
%%% Unten: Links, Mitte, Rechts
 \ifoot[]{\vspace{-3pt}\href{http://ansible.github.com/}{ansible.github.com}}
 \cfoot[]{}
 \ofoot[]{\vspace{-3pt}\copyright 2012 \href{mailto:jpmens@gmail.com}{Jan-Piet Mens}}
%%% }}}
%%%%%%%%%%%%%%%%%%%%%%%%%%%%%%%%%%%%%%%%%%%%%%%%%%%%%%%%%%%%

%%%%%%%%%%%%%%%%%%%%%%%%%%%%%%%%%%%%%%%%%%%%%%%%%%%%%%%%%%%%
%%% Sonstiges {{{
% \setlength{\parindent}{17pt}      % Einzug 17pt,
% \setlength{\parskip}{2pt}         % keine Leerzeilen.

% \textwidth      127mm             % Textbreite
% \textheight     235mm             % Texth�he
% \topmargin     -5mm               % Abstand oben
% \oddsidemargin  7mm               % Abstand Links, onepage

%\onehalfspacing                    % Zeilenabstand: Bei korrektur,
 \singlespacing                     % bei Abgabe

% Punkt- und Komma Abst�nde bei Tausendern/
% Dezimalzahlen ans deutsche anpassen!
 \mathcode`,="013B
 \mathcode`.="613A

 \setlength{\emergencystretch}{2em} % Notfallsstreckung
%%% }}}
%%%%%%%%%%%%%%%%%%%%%%%%%%%%%%%%%%%%%%%%%%%%%%%%%%%%%%%%%%%%

\addtolength{\voffset}{10pt}
%\renewcommand{\figurename}{Abb.}
\begin{document}

%%%%%%%%%%%%%%%%%%%%%%%%%%%%%%%%%%%%%%%%%%%%%%%%%%%%%%%%%%%%
%%% Titelseite {{{
%\pagenumbering{none}                % F�r die Titelseite: Keine Seitennummern,
%\thispagestyle{empty}               % keine Kopf- und Fu�zeilen.
%
%\begin{center}
%
%\end{center}
%\newpage
%
%%% }}}
%%%%%%%%%%%%%%%%%%%%%%%%%%%%%%%%%%%%%%%%%%%%%%%%%%%%%%%%%%%%

%%%%%%%%%%%%%%%%%%%%%%%%%%%%%%%%%%%%%%%%%%%%%%%%%%%%%%%%%%%%
%%% Inhaltsverzeichnis {{{
  \pagenumbering{arabic}            % Arabische Nummerierung
% \pagenumbering{roman}             % Kleine, r�mische Nummerierung
% \tableofcontents                  % Das Inhaltsverzeichnis
% \listoffigures                    % Verzeichnis aller Abbildungen
% \listoftables                     % Verzeichnis aller Tabellen
% \pagenumbering{arabic}            % ...und wieder Arabisch
% \newpage
%%% }}}
%%%%%%%%%%%%%%%%%%%%%%%%%%%%%%%%%%%%%%%%%%%%%%%%%%%%%%%%%%%%

% we won't be using math mode much, so redefine some of the characters
% we might want to talk about
\catcode`\^=12
\catcode`\_=12
%
\chardef\\=`\\
\chardef\{=`\{
\chardef\}=`\}

\parskip=0pt
\setlength{\tabcolsep}{0pt}

%%%%%%%%%%%%%%%%%%%%%%%%%%%%%%%%%%%%%%%%%%%%%%%%%%%%%%%%%%%%
%%% Inhalt {{{

%Schon in lhead
%\section{GRML's Zsh-Setup Reference Card}

\subsection*{Ansible}

Ansible, created by  Michael DeHaan, is a radically simple model-driven configuration management, deployment, and command execution framework. Other than Python 2.6 and a working SSH infrastructure, Ansible requires no setup, no daemons, no PKI, no nothing (Fig.~\ref{ansiblearchitecture}). 

\bild{250pt}{art/ansible.png}{Ansible architecture}{ansiblearchitecture}

\subsection*{Problems?}

If you need help or want to report a problem Ansible has a mailing-list
at \href{http://groups.google.com/group/ansible-project}{groups.google.com/group/ansible-project}. The latest and greatest code is on \href{https://github.com/ansible/ansible}{github.com/ansible/ansible}
where you can track issues and submit ideas. There are also lots of people
willing to assist on IRC: log in to \C{\#ansible} on FreeNode.

\subsection*{Getting started}

In this document we'll call the machine you run Ansible on (i.e. the machine from which you deploy) the \master{}; machines onto which you deploy (i.e. your clients), we'll call \nodes{}. On the \master{} you don't necessarily require \C{root} permission: Ansible can run as any user. Similarly, on \nodes{} you don't need root permissions either (well, maybe not): Ansible can connect to \nodes{} as a "normal" user and then (if required) \C{sudo} to root.

Download Ansible and unpack. You don't need to install it ... running from a git checkout is fine.
Create your \I{inventory} file and test: See the section on SSH for setting up SSH.

Ansible requires Python 2.6 though nodes only 2.4



\begin{small}
\begin{verbatim}
\end{verbatim}
\end{small}

\subsection*{Inventory}

\small{Ansible works against multiple systems in your infrastructure at the same time. It does this by selecting portions of systems listed in the inventory file, which defaults to \C{/etc/ansible/hosts}.}

\begin{extymeta}
localhost

[webs]
www.example.com
web[09-12].example.com
192.168.8.9

[devservers]
box1.example.com
jo.example.com \B{ntpserver=127.0.0.1}
\end{extymeta}

That last entry has what is called a \I{host var}; ignore that for now.


\subsection*{Modules}

Ansible ships with a number of modules (called the "module library") that can be executed directly on remote hosts or through Playbooks. Users can also write their own modules. These modules can control system resources, like services, packages, or files (anything really), or handle executing system commands. The following is a list of modules in the core library with supported options (\man{} is mandatory, \opt{} optional). (If you're viewing this in a PDF reader, click on the module name to go to the official module documentation at \href{http://ansible.github.com/}{ansible.github.com}.)

\begin{extymeta}
ansible 192.168.8.9 -m \B{ping}
ansible webs -m \B{copy} -a "src=/tmp/f dest=/etc/conf"
\end{extymeta}

%-----------------------------------------------------------------------

% modules2.py ---> /Users/jpm/Auto/pubgit/ansible/ansible/library/apt
% modules2.py ---> /Users/jpm/Auto/pubgit/ansible/ansible/library/apt_repository
% modules2.py ---> /Users/jpm/Auto/pubgit/ansible/ansible/library/assemble
% modules2.py ---> /Users/jpm/Auto/pubgit/ansible/ansible/library/async_status
% modules2.py ---> /Users/jpm/Auto/pubgit/ansible/ansible/library/async_wrapper
% modules2.py ---> /Users/jpm/Auto/pubgit/ansible/ansible/library/authorized_key
% modules2.py ---> /Users/jpm/Auto/pubgit/ansible/ansible/library/command
% modules2.py ---> /Users/jpm/Auto/pubgit/ansible/ansible/library/copy
% modules2.py ---> /Users/jpm/Auto/pubgit/ansible/ansible/library/easy_install
% modules2.py ---> /Users/jpm/Auto/pubgit/ansible/ansible/library/facter
% modules2.py ---> /Users/jpm/Auto/pubgit/ansible/ansible/library/fetch
% modules2.py ---> /Users/jpm/Auto/pubgit/ansible/ansible/library/file
%--- FILE  ----  from /Users/jpm/Auto/pubgit/ansible/ansible/library/file ---

%: -- module header
\mods{file}{file}{
		Sets attributes of files, symlinks, and directories, or removes files/symlinks/directories. Many other modules support the same options as the file module - including \M{copy}, \M{template}, and \M{assmeble}. 
		
		}

%: -- module options

\begin{xlist}{abcdefghijklmno}
  \item[\man\,\C{dest}]
	defines the file being managed, unless when used with \I{state=link}, and then sets the destination to create a symbolic link to using \I{src} 
		
		\item[\opt\,\C{state}]
	If directory, all immediate subdirectories will be created if they do not exist. If \I{file}, the file will NOT be created if it does not exist, see the \M{copy} or \M{template} module if you want that behavior. If \I{link}, the symbolic link will be created or changed. If absent, directories will be recursively deleted, and files or symlinks will be unlinked. 
		
		\B{Choices}:\,
		    \C{file},
		    \C{link},
		    \C{directory},
		    \C{absent}.
		    (default \C{file})
		\item[\opt\,\C{mode}]
	mode the file or directory should be, such as 0644 as would be fed to \I{chmod}. English modes like \B{g+x} are not yet supported 
		
		\end{xlist}
\I{See also \M{copy}, \M{template}, \M{assemble}} 
		



% modules2.py ---> /Users/jpm/Auto/pubgit/ansible/ansible/library/get_url
%--- GET_URL  ----  from /Users/jpm/Auto/pubgit/ansible/ansible/library/get_url ---

%: -- module header
\mods{get_url}{get-url}{
		Downloads files from HTTP, HTTPS, or FTP to the remote server. The remote server must have direct access to the remote resource. 
		
		}

%: -- module options

\begin{xlist}{abcdefghijklmno}
  \item[\man\,\C{url}]
	HTTP, HTTPS, or FTP URL 
		
		\item[\man\,\C{dest}]
	absolute path of where to download the file to. 
		If \I{dest} is a directory, the basename of the file on the remote server will be used. If a directory, \I{thirsty=yes} must also be set. 
		
		\item[\opt\,\C{thirsty}]
	if \C{yes}, will download the file every time and replace the file if the contents change. if \C{no}, the file will only be downloaded if the destination does not exist. Generally should be \C{yes} only for small local files. prior to 0.6, acts if \C{yes} by default. 
		
		\B{Choices}:\,
		    \C{yes},
		    \C{no}.
		    (default \C{no})
		(\I{* version 0.7})
		\item[\opt\,\C{others}]
	all arguments accepted by the \M{file} module also work here 
		
		\end{xlist}
\I{This module doesn't support proxies or passwords.} 
		\I{Also see the \M{template} module.} 
		



% modules2.py ---> /Users/jpm/Auto/pubgit/ansible/ansible/library/git
% modules2.py ---> /Users/jpm/Auto/pubgit/ansible/ansible/library/group
% modules2.py ---> /Users/jpm/Auto/pubgit/ansible/ansible/library/jpfact
% modules2.py ---> /Users/jpm/Auto/pubgit/ansible/ansible/library/lineinfile
% modules2.py ---> /Users/jpm/Auto/pubgit/ansible/ansible/library/mount
% modules2.py ---> /Users/jpm/Auto/pubgit/ansible/ansible/library/mysql_db
% modules2.py ---> /Users/jpm/Auto/pubgit/ansible/ansible/library/mysql_user
% modules2.py ---> /Users/jpm/Auto/pubgit/ansible/ansible/library/nagios
% modules2.py ---> /Users/jpm/Auto/pubgit/ansible/ansible/library/ohai
% modules2.py ---> /Users/jpm/Auto/pubgit/ansible/ansible/library/ping
% modules2.py ---> /Users/jpm/Auto/pubgit/ansible/ansible/library/pip
% modules2.py ---> /Users/jpm/Auto/pubgit/ansible/ansible/library/postgresql_db
% modules2.py ---> /Users/jpm/Auto/pubgit/ansible/ansible/library/postgresql_user
% modules2.py ---> /Users/jpm/Auto/pubgit/ansible/ansible/library/raw
%--- RAW  ----  from /Users/jpm/Auto/pubgit/ansible/ansible/library/raw ---

%: -- module header
\mods{raw}{raw}{
		Executes a low-down and dirty SSH command, not going through the module subsystem. This is useful and should only be done in two cases. The first case is installing python-simplejson on older (Python 2.4 and before) hosts that need it as a dependency to run modules, since nearly all core modules require it. Another is speaking to any devices such as routers that do not have any Python installed. In any other case, using the \M{shell} or \M{command} module is much more appropriate. Arguments given to \M{raw} are run directly through the configured remote shell and only output is returned. There is no error detection or change handler support for this module 
		
		}

%: -- module options



%--- BEGIN-EXTRADATA
\begin{extymeta}
ansible host -m \B{raw} -a "yum -y install python-simplejson"
\end{extymeta}
%----- END-EXTRADATA

% modules2.py ---> /Users/jpm/Auto/pubgit/ansible/ansible/library/seboolean
% modules2.py ---> /Users/jpm/Auto/pubgit/ansible/ansible/library/selinux
% modules2.py ---> /Users/jpm/Auto/pubgit/ansible/ansible/library/service
% modules2.py ---> /Users/jpm/Auto/pubgit/ansible/ansible/library/setup
%--- SETUP  ----  from /Users/jpm/Auto/pubgit/ansible/ansible/library/setup ---

%: -- module header
\mods{setup}{setup}{
		This module is automatically called by playbooks to gather useful variables about remote hosts that can be used in playbooks. It can also be executed directly by \C{/usr/bin/ansible} to check what variables are available to a host. Ansible provides many \I{facts} about the system, automatically. 
		
		}

%: -- module options

\I{More ansible facts will be added with successive releases. If \I{facter} or \I{ohai} are installed, variables from these programs will also be snapshotted into the JSON file for usage in templating. These variables are prefixed with \C{facter_} and \C{ohai_} so it's easy to tell their source. All variables are bubbled up to the caller. Using the ansible facts and choosing to not install \I{facter} and \I{ohai} means you can avoid Ruby-dependencies on your remote systems.} 
		


%--- BEGIN-EXTRADATA
\begin{extymeta}

"ansible_architecture": "x86_64",
"ansible_distribution": "CentOS",
"ansible_distribution_release": "Final",
"ansible_distribution_version": "6.2",
"ansible_eth0": \{
    "ipv4": \{
        "address": "REDACTED",
        "netmask": "255.255.255.0"
    \},
    "ipv6": [
        \{
            "address": "REDACTED",
            "prefix": "64",
            "scope": "link"
        \}
    ],
    "macaddress": "REDACTED"
\},
"ansible_form_factor": "Other",
"ansible_fqdn": "localhost.localdomain",
"ansible_hostname": "localhost",
"ansible_interfaces": [
    "lo",
    "eth0"
],
\end{extymeta}
%----- END-EXTRADATA

% modules2.py ---> /Users/jpm/Auto/pubgit/ansible/ansible/library/shell
% modules2.py ---> /Users/jpm/Auto/pubgit/ansible/ansible/library/slurp
% modules2.py ---> /Users/jpm/Auto/pubgit/ansible/ansible/library/sshfp
% modules2.py ---> /Users/jpm/Auto/pubgit/ansible/ansible/library/subversion
% modules2.py ---> /Users/jpm/Auto/pubgit/ansible/ansible/library/supervisorctl
% modules2.py ---> /Users/jpm/Auto/pubgit/ansible/ansible/library/template
% modules2.py ---> /Users/jpm/Auto/pubgit/ansible/ansible/library/user
% modules2.py ---> /Users/jpm/Auto/pubgit/ansible/ansible/library/virt
% modules2.py ---> /Users/jpm/Auto/pubgit/ansible/ansible/library/wait_for
% modules2.py ---> /Users/jpm/Auto/pubgit/ansible/ansible/library/yum


%-----------------------------------------------------------------------
\subsection*{Playbooks}

Simply put, \href{http://ansible.github.com/playbooks.html}{Playbooks} are the
basis for a really simple configuration management and multi-machine deployment
system, unlike any that already exist, and one that is very well suited to
deploying complex applications. Playbooks can declare configurations, but they
can also orchestrate steps of any manual ordered process, even as different
steps must bounce back and forth between sets of machines in particular orders.
They can launch tasks synchronously or asynchronously. Playbooks are expressed
in YAML\footnote{\url{http://www.yaml.org/}} format and have a minimum of syntax. (Tip: use the \I{Online YAML Parser}\footnote{\url{http://yaml-online-parser.appspot.com/}} to experiment.)
Each playbook is composed of one
or more \I{plays} in a list. Here's a playbook that contains just one play:

\begin{extymeta}
---
- hosts: devservers
  vars:
    http\_port: 80
    \B{conf}: httpd.j2
  user: root
  tasks:
  - name: ensure apache is at the latest version
    action: yum pkg=httpd state=latest
  - name: write the apache config file
    action: template src=/srv/\B{\$\{conf\}} dest=/etc/httpd.conf
    notify:
    - restart apache
  - name: ensure apache is running
    action: service name=httpd state=started
  handlers:
    - name: restart apache
      action: service name=apache state=restarted
\end{extymeta}

(You can omit a task's \I{name} if you ensure the task's \I{action} becomes the key.)

\begin{extymeta}
ansible-playbook -u jpm mini.yaml
\end{extymeta}

Use the \C{--verbose} flag for more information.


\begin{extymeta}
only_if: "not '${ansible_cmdline.BOOT_IMAGE}'.startswith('$')"
\end{extymeta}


\subsection*{Handlers \& Notification}

Modules are written to be \I{idempotent} and can relay when they have made a
change on the remote system. Playbooks recognize this and have a basic event
system that can be used to respond to change.  These \I{notify} actions are
triggered at the end of each \I{play} in a playbook, and trigger only once
each. For instance, multiple resources may indicate that apache needs to be
restarted, but apache will only be bounced once. Here's an example of
restarting two services when the contents of a file change, but only if the
file changes:

\begin{extymeta}
- name: template configuration file
  action: template src=template.j2 dest=/etc/foo.conf
  \B{notify}:
     - restart memcached
     - restart apache
\end{extymeta}

The things listed in the \I{notify} section of a task are called \I{handlers}. Handlers are lists of tasks, not really any different from regular tasks, that are referenced by name. Handlers are what notifiers notify. If nothing notifies a handler, it will not run. Regardless of how many things notify a handler, it will run only once, after all of the tasks complete in a particular play. Handlers are best used to restart services and trigger reboots. You probably won�t need them for much else. Here�s an example handlers section:

\begin{extymeta}
handlers:
    - name: restart memcached
      action: service name=memcached state=restarted
    - name: restart apache
      action: service name=apache state=restarted
\end{extymeta}

Notify handlers are always run in the order written.

%-----------------------------------------------------------------------


%-----------------------------------------------------------------------
\subsection*{Templates}

FIXME: short description of Jinja2 templates with one or two short examples using some vars from playbook and \M{setup}.

show:
        expansion
	        if, else, endif
		        switch / case ????
			        loops



\subsection*{Delegation}

FIXME: what delegation is. mention localaction needs SSH to localhost

\subsection*{Facts}

refer to \M{setup}; maybe show short module example?


\bild{250pt}{art/ansible-facts-local.png}{That's a foo\ldots}{factslocal}

\subsection*{Variables}

\subsubsection*{Host vars}

\subsubsection*{Group vars}


\subsection*{SSH}

Paramiko is a native python implementation of the SSH protocol. It's
the default transport method used by Ansible, this method should work
for 99\% of people by default. The native ssh transport method is
necessary in more complicated infrastructures where support for
bastion ('proxy') hosts is required, or if GSSAPI (kerberos) is used
for authentication. The native ssh transport will recognize your
\C{~/.ssh/config} file because it uses the 'ssh' command installed on the
local system.

[somegroup]
foo ansible_ssh_port=1234
bar ansible_ssh_port=1235

\subsection*{Shell variables used by Ansible}


You might want to stick to
just mentioning \$ANSIBLE_CONFIG and giving reference to
http://ansible.github.com/examples.html\#configuration-defaults


\begin{xlist}{ABCDEFGHIGJKLMNOP}

	\item[ANSIBLE\_SSH\_ARGS]

	\item[ANSIBLE\_REMOTE\_USER]

\end{xlist}

%-----------------------------------------------------------------------
\subsection*{Extending Ansible}

\ansible{} is extensible: you can use the \ansible{} Python API to control
nodes, you can extend \ansible{} to respond to various python events, and you can
plug in inventory data from external data sources. \ansible{} is written in its
own API so you have a considerable amount of power across the
board\footnote{\url{http://ansible.github.com/api.html}}.

\subsubsection*{Your own modules}

\ansible{} modules are reusable units which can be used by the \ansible{} API,
or by the \C{ansible} or \C{ansible-playbook} programs. Modules can be written
in any language supported by \I{nodes} (e.g. shell
scripts\footnote{\url{http://mens.de/:/ansshell}}) and are found in the path
specified by \C{ANSIBLE_LIBRARY_PATH} or the \C{--module-path} command-line
option\footnote{\url{http://ansible.github.com/moduledev.html}}.

The following listing illustrates what an \ansible{} module looks like in Python;
it accepts a single parameter (\C{name}) with the name of a file on a node for
which the file size should be retrieved:

\begin{extymeta}
#!/usr/bin/python
import os

DOCUMENTATION = '''
---
module: mini
short_description: Determine file size of remote file
description:
     - Determines the size of a specified file.
version_added: "0.0"
options:
  - name:
      description:
        - Absolute path to the file name on the remote node.
      required: true
      default: null
      aliases: [dest, destfile]
'''

def main():
    module = AnsibleModule(
        argument_spec = dict(
            name=dict(required=True, aliases=['dest', 'destfile']),
        ),
    )

    params = module.params
    filename = params['name']

    try:
        stat = os.stat(filename)
    except:
        module.fail_json(msg="Can't stat file: %s" % filename)

    changed = False
    msg = "Filename %s has size %s" % (filename, stat.st_size)
    module.exit_json(changed=changed, msg=msg)

# this is magic, see lib/ansible/module_common.py
#<<INCLUDE_ANSIBLE_MODULE_COMMON>>

main()
\end{extymeta}

Use this module in a playbook or from the command-line:

\begin{extymeta}
ansible 127.0.0.1 -c local \B{-m mini} -a dest=/tmp/xx
127.0.0.1 | success >> \{
    "changed": false, 
    "msg": "Filename /tmp/xx has size 1233"
\}
\end{extymeta}

You should look at some of the modules in \ansible{}'s \C{library/*} for
inspiration before writing your own.

Modules you write can also return \I{facts} like the \M{setup} module does, but
you have to call your modules explicitly, whereas \M{setup} is invoked
automatically from a playbook. (Fig.~\ref{factslocal}) A fact-gathering module
can be written as trivially as this example in a shell script\footnote{\url{}}.

\begin{extymeta}
#!/bin/sh

COUNT=`who | wc -l`
cat <<EOF
\{
   "ansible_facts" : \{
      "users_logged_in" : $COUNT
    \}
\}
EOF
\end{extymeta}


\subsubsection*{Callback plugins}

\subsubsection*{Action plugins}


%-----------------------------------------------------------------------
\subsection*{Pull mode}

Instead of pushing configuration from a master to nodes it may be advantageous to pull from the master onto the nodes.
\ansible{} can run a playbook on a node if the node has a full installation of \ansible{} and its dependencies.
The example in Fig.~\ref{ansiblepullmode} shows how to accomplish this using a repository available to all nodes from which they obtain current copies of Playbooks and supporting files. \href{http://jpmens.net/2012/07/14/ansible-pull-instead-of-push/}{Read more}.

\bild{250pt}{art/ansible-pull-nodes.png}{Ansible pull-mode}{ansiblepullmode}



\subsection*{Tips and tricks: fun with Ansible}

\begin{enumerate}

    \item[1.]

	If your \I{nodes} have a version of Python which doesn't meet Ansible's
	requirements, install Python non-destructively in, say,
	\C{/usr/local/Python}, and configure your inventory to use that path on
	nodes (e.g. with a group variable).

	\begin{extymeta}
	ansible\_python\_interpreter: /usr/local/Python
	\end{extymeta}

    \item[2.]

    	Install \href{http://en.wikipedia.org/wiki/Cowsay}{Cowsay}. You must! (And make
	sure it's (symlinked) in \C{/usr/bin/cowsay}.)

	\begin{extymeta}
	 _____________________________ 
	< TASK: [Ansible says mooooo] >
	 ----------------------------- 
	        \\   ^__^
	         \\  (oo)\\_______
	            (__)\\       )\\/\\
	                ||----w |
	                ||     ||
	\end{extymeta}

    \item[3.]
    	Other "hidden" ansible variables FIXME

\end{enumerate}


%%%%%%%%%%%%%%%%%%%%%%%%%%%%%%%%%%%%%%%%%%%%%%%%%%%%%%%%%%%%%%%%%%%%%%
% Dieser Trenner muss eingef�gt werden, wenn eine Tabelle zu lang ist
% und daher nicht von LaTeX umbrochen wird.
%\commandlistend
%
%\commandlistbegin
%%%%%%%%%%%%%%%%%%%%%%%%%%%%%%%%%%%%%%%%%%%%%%%%%%%%%%%%%%%%%%%%%%%%%%


%%% }}}
%%%%%%%%%%%%%%%%%%%%%%%%%%%%%%%%%%%%%%%%%%%%%%%%%%%%%%%%%%%%


\end{document}

%%% vim:set ai tw=80 fdm=marker:   EOF
