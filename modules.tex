% modules2.py ---> /Users/jpm/Auto/pubgit/ansible/ansible/library/apt
% modules2.py ---> /Users/jpm/Auto/pubgit/ansible/ansible/library/apt_repository
% modules2.py ---> /Users/jpm/Auto/pubgit/ansible/ansible/library/assemble
% modules2.py ---> /Users/jpm/Auto/pubgit/ansible/ansible/library/async_status
% modules2.py ---> /Users/jpm/Auto/pubgit/ansible/ansible/library/async_wrapper
% modules2.py ---> /Users/jpm/Auto/pubgit/ansible/ansible/library/authorized_key
% modules2.py ---> /Users/jpm/Auto/pubgit/ansible/ansible/library/command
% modules2.py ---> /Users/jpm/Auto/pubgit/ansible/ansible/library/copy
% modules2.py ---> /Users/jpm/Auto/pubgit/ansible/ansible/library/easy_install
% modules2.py ---> /Users/jpm/Auto/pubgit/ansible/ansible/library/facter
% modules2.py ---> /Users/jpm/Auto/pubgit/ansible/ansible/library/fetch
% modules2.py ---> /Users/jpm/Auto/pubgit/ansible/ansible/library/file
%--- FILE  ----  from /Users/jpm/Auto/pubgit/ansible/ansible/library/file ---

%: -- module header
\mods{file}{file}{
		Sets attributes of files, symlinks, and directories, or removes files/symlinks/directories. Many other modules support the same options as the file module - including \M{copy}, \M{template}, and \M{assmeble}. 
		
		}

%: -- module options

\begin{xlist}{abcdefghijklmno}
  \item[\man\,\C{dest}]
	defines the file being managed, unless when used with \I{state=link}, and then sets the destination to create a symbolic link to using \I{src} 
		
		\item[\opt\,\C{state}]
	If directory, all immediate subdirectories will be created if they do not exist. If \I{file}, the file will NOT be created if it does not exist, see the \M{copy} or \M{template} module if you want that behavior. If \I{link}, the symbolic link will be created or changed. If absent, directories will be recursively deleted, and files or symlinks will be unlinked. 
		
		\B{Choices}:\,
		    \C{file},
		    \C{link},
		    \C{directory},
		    \C{absent}.
		    (default \C{file})
		\item[\opt\,\C{mode}]
	mode the file or directory should be, such as 0644 as would be fed to \I{chmod}. English modes like \B{g+x} are not yet supported 
		
		\end{xlist}
\I{See also \M{copy}, \M{template}, \M{assemble}} 
		



% modules2.py ---> /Users/jpm/Auto/pubgit/ansible/ansible/library/get_url
%--- GET_URL  ----  from /Users/jpm/Auto/pubgit/ansible/ansible/library/get_url ---

%: -- module header
\mods{get_url}{get-url}{
		Downloads files from HTTP, HTTPS, or FTP to the remote server. The remote server must have direct access to the remote resource. 
		
		}

%: -- module options

\begin{xlist}{abcdefghijklmno}
  \item[\man\,\C{url}]
	HTTP, HTTPS, or FTP URL 
		
		\item[\man\,\C{dest}]
	absolute path of where to download the file to. 
		If \I{dest} is a directory, the basename of the file on the remote server will be used. If a directory, \I{thirsty=yes} must also be set. 
		
		\item[\opt\,\C{thirsty}]
	if \C{yes}, will download the file every time and replace the file if the contents change. if \C{no}, the file will only be downloaded if the destination does not exist. Generally should be \C{yes} only for small local files. prior to 0.6, acts if \C{yes} by default. 
		
		\B{Choices}:\,
		    \C{yes},
		    \C{no}.
		    (default \C{no})
		(\I{* version 0.7})
		\item[\opt\,\C{others}]
	all arguments accepted by the \M{file} module also work here 
		
		\end{xlist}
\I{This module doesn't support proxies or passwords.} 
		\I{Also see the \M{template} module.} 
		



% modules2.py ---> /Users/jpm/Auto/pubgit/ansible/ansible/library/git
% modules2.py ---> /Users/jpm/Auto/pubgit/ansible/ansible/library/group
% modules2.py ---> /Users/jpm/Auto/pubgit/ansible/ansible/library/jpfact
% modules2.py ---> /Users/jpm/Auto/pubgit/ansible/ansible/library/lineinfile
% modules2.py ---> /Users/jpm/Auto/pubgit/ansible/ansible/library/mount
% modules2.py ---> /Users/jpm/Auto/pubgit/ansible/ansible/library/mysql_db
% modules2.py ---> /Users/jpm/Auto/pubgit/ansible/ansible/library/mysql_user
% modules2.py ---> /Users/jpm/Auto/pubgit/ansible/ansible/library/nagios
% modules2.py ---> /Users/jpm/Auto/pubgit/ansible/ansible/library/ohai
% modules2.py ---> /Users/jpm/Auto/pubgit/ansible/ansible/library/ping
% modules2.py ---> /Users/jpm/Auto/pubgit/ansible/ansible/library/pip
% modules2.py ---> /Users/jpm/Auto/pubgit/ansible/ansible/library/postgresql_db
% modules2.py ---> /Users/jpm/Auto/pubgit/ansible/ansible/library/postgresql_user
% modules2.py ---> /Users/jpm/Auto/pubgit/ansible/ansible/library/raw
%--- RAW  ----  from /Users/jpm/Auto/pubgit/ansible/ansible/library/raw ---

%: -- module header
\mods{raw}{raw}{
		Executes a low-down and dirty SSH command, not going through the module subsystem. This is useful and should only be done in two cases. The first case is installing python-simplejson on older (Python 2.4 and before) hosts that need it as a dependency to run modules, since nearly all core modules require it. Another is speaking to any devices such as routers that do not have any Python installed. In any other case, using the \M{shell} or \M{command} module is much more appropriate. Arguments given to \M{raw} are run directly through the configured remote shell and only output is returned. There is no error detection or change handler support for this module 
		
		}

%: -- module options



%--- BEGIN-EXTRADATA
\begin{extymeta}
ansible host -m \B{raw} -a "yum -y install python-simplejson"
\end{extymeta}
%----- END-EXTRADATA

% modules2.py ---> /Users/jpm/Auto/pubgit/ansible/ansible/library/seboolean
% modules2.py ---> /Users/jpm/Auto/pubgit/ansible/ansible/library/selinux
% modules2.py ---> /Users/jpm/Auto/pubgit/ansible/ansible/library/service
% modules2.py ---> /Users/jpm/Auto/pubgit/ansible/ansible/library/setup
%--- SETUP  ----  from /Users/jpm/Auto/pubgit/ansible/ansible/library/setup ---

%: -- module header
\mods{setup}{setup}{
		This module is automatically called by playbooks to gather useful variables about remote hosts that can be used in playbooks. It can also be executed directly by \C{/usr/bin/ansible} to check what variables are available to a host. Ansible provides many \I{facts} about the system, automatically. 
		
		}

%: -- module options

\I{More ansible facts will be added with successive releases. If \I{facter} or \I{ohai} are installed, variables from these programs will also be snapshotted into the JSON file for usage in templating. These variables are prefixed with \C{facter_} and \C{ohai_} so it's easy to tell their source. All variables are bubbled up to the caller. Using the ansible facts and choosing to not install \I{facter} and \I{ohai} means you can avoid Ruby-dependencies on your remote systems.} 
		


%--- BEGIN-EXTRADATA
\begin{extymeta}

"ansible_architecture": "x86_64",
"ansible_distribution": "CentOS",
"ansible_distribution_release": "Final",
"ansible_distribution_version": "6.2",
"ansible_eth0": \{
    "ipv4": \{
        "address": "REDACTED",
        "netmask": "255.255.255.0"
    \},
    "ipv6": [
        \{
            "address": "REDACTED",
            "prefix": "64",
            "scope": "link"
        \}
    ],
    "macaddress": "REDACTED"
\},
"ansible_form_factor": "Other",
"ansible_fqdn": "localhost.localdomain",
"ansible_hostname": "localhost",
"ansible_interfaces": [
    "lo",
    "eth0"
],
\end{extymeta}
%----- END-EXTRADATA

% modules2.py ---> /Users/jpm/Auto/pubgit/ansible/ansible/library/shell
% modules2.py ---> /Users/jpm/Auto/pubgit/ansible/ansible/library/slurp
% modules2.py ---> /Users/jpm/Auto/pubgit/ansible/ansible/library/sshfp
% modules2.py ---> /Users/jpm/Auto/pubgit/ansible/ansible/library/subversion
% modules2.py ---> /Users/jpm/Auto/pubgit/ansible/ansible/library/supervisorctl
% modules2.py ---> /Users/jpm/Auto/pubgit/ansible/ansible/library/template
% modules2.py ---> /Users/jpm/Auto/pubgit/ansible/ansible/library/user
% modules2.py ---> /Users/jpm/Auto/pubgit/ansible/ansible/library/virt
% modules2.py ---> /Users/jpm/Auto/pubgit/ansible/ansible/library/wait_for
% modules2.py ---> /Users/jpm/Auto/pubgit/ansible/ansible/library/yum
