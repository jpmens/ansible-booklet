\subsection*{Playbooks}

Simply put, \href{http://ansible.github.com/playbooks.html}{Playbooks} are the
basis for a really simple configuration management and multi-machine deployment
system, unlike any that already exist, and one that is very well suited to
deploying complex applications. Playbooks can declare configurations, but they
can also orchestrate steps of any manual ordered process, even as different
steps must bounce back and forth between sets of machines in particular orders.
They can launch tasks synchronously or asynchronously. Playbooks are expressed
in YAML\footnote{\url{http://www.yaml.org/}} format and have a minimum of syntax. (Tip: use the \I{Online YAML Parser}\footnote{\url{http://yaml-online-parser.appspot.com/}} to experiment.)
Each playbook is composed of one
or more \I{plays} in a list. Here's a playbook that contains just one play:

\begin{extymeta}
---
- hosts: devservers
  vars:
    http\_port: 80
    \B{conf}: httpd.j2
  user: root
  tasks:
  - name: ensure apache is at the latest version
    action: yum pkg=httpd state=latest
  - name: write the apache config file
    action: template src=/srv/\B{\$\{conf\}} dest=/etc/httpd.conf
    notify:
    - restart apache
  - name: ensure apache is running
    action: service name=httpd state=started
  handlers:
    - name: restart apache
      action: service name=apache state=restarted
\end{extymeta}

\begin{extymeta}
ansible-playbook -u jpm mini.yaml
\end{extymeta}

Use the \C{--verbose} flag for more information.


\begin{extymeta}
only_if: "not '${ansible_cmdline.BOOT_IMAGE}'.startswith('$')"
\end{extymeta}


\subsection*{Handlers \& Notification}

Modules are written to be \I{idempotent} and can relay when they have made a
change on the remote system. Playbooks recognize this and have a basic event
system that can be used to respond to change.  These \I{notify} actions are
triggered at the end of each \I{play} in a playbook, and trigger only once
each. For instance, multiple resources may indicate that apache needs to be
restarted, but apache will only be bounced once. Here's an example of
restarting two services when the contents of a file change, but only if the
file changes:

\begin{extymeta}
- name: template configuration file
  action: template src=template.j2 dest=/etc/foo.conf
  \B{notify}:
     - restart memcached
     - restart apache
\end{extymeta}

The things listed in the \I{notify} section of a task are called \I{handlers}. Handlers are lists of tasks, not really any different from regular tasks, that are referenced by name. Handlers are what notifiers notify. If nothing notifies a handler, it will not run. Regardless of how many things notify a handler, it will run only once, after all of the tasks complete in a particular play. Handlers are best used to restart services and trigger reboots. You probably won�t need them for much else. Here�s an example handlers section:

\begin{extymeta}
handlers:
    - name: restart memcached
      action: service name=memcached state=restarted
    - name: restart apache
      action: service name=apache state=restarted
\end{extymeta}

Notify handlers are always run in the order written.

%-----------------------------------------------------------------------
