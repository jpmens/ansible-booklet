\subsection*{Extending Ansible}

\ansible{} is extensible: you can use the \ansible{} Python API to control
nodes, you can extend Ansible to respond to various python events, and you can
plug in inventory data from external data sources. \ansible{} is written in its
own API so you have a considerable amount of power across the
board\footnote{\url{http://ansible.github.com/api.html}}.

\subsubsection*{Your own modules}

\ansible{} modules are reusable units which can be used by the \ansible{} API,
or by the \C{ansible} or \C{ansible-playbook} programs. Modules can be written
in any language supported by \I{nodes} (e.g. shell
scripts\footnote{\url{http://mens.de/:/ansshell}}) and are found in the path
specified by \C{ANSIBLE_LIBRARY_PATH} or the \C{--module-path} command-line
option\footnote{\url{http://ansible.github.com/moduledev.html}}.

The following listing illustrates what an \ansible{} module looks like in Python;
it accepts a single parameter (\C{name}) with the name of a file on a node for
which the file size should be retrieved:

\begin{extymeta}
#!/usr/bin/python
import os

DOCUMENTATION = '''
---
module: mini
short_description: Determine file size of remote file
description:
     - Determines the size of a specified file.
version_added: "0.0"
options:
  - name:
      description:
        - Absolute path to the file name on the remote node.
      required: true
      default: null
      aliases: [dest, destfile]
'''

def main():
    module = AnsibleModule(
        argument_spec = dict(
            name=dict(required=True, aliases=['dest', 'destfile']),
        ),
    )

    params = module.params
    filename = params['name']

    try:
        stat = os.stat(filename)
    except:
        module.fail_json(msg="Can't stat file: %s" % filename)

    changed = False
    msg = "Filename %s has size %s" % (filename, stat.st_size)
    module.exit_json(changed=changed, msg=msg)

# this is magic, see lib/ansible/module_common.py
#<<INCLUDE_ANSIBLE_MODULE_COMMON>>

main()
\end{extymeta}

Use this module in a playbook or from the command-line:

\begin{extymeta}
ansible 127.0.0.1 -c local \B{-m mini} -a dest=/tmp/xx
127.0.0.1 | success >> \{
    "changed": false, 
    "msg": "Filename /tmp/xx has size 1233"
\}
\end{extymeta}

You should look at some of the modules in \ansible{}'s \C{library/*} for
inspiration before writing your own.

Modules you write can also return \I{facts} like the \M{setup} module does, but
you have to call your modules explicitly, whereas \M{setup} is invoked
automatically from a playbook. (Fig.~\ref{factslocal}) A fact-gathering module
can be written as trivially as this example in a shell script\footnote{\url{}}.

\begin{extymeta}
#!/bin/sh

COUNT=`who | wc -l`
cat <<EOF
\{
   "ansible_facts" : \{
      "users_logged_in" : $COUNT
    \}
\}
EOF
\end{extymeta}


\bild{250pt}{art/ansible-facts-local.png}{Ansible copies modules to nodes}{factslocal}

\subsubsection*{Callback plugins}

\subsubsection*{Action plugins}
